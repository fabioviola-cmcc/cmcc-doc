\documentclass[twocolumn,8pt]{article}
\usepackage[utf8]{inputenc}
\usepackage{hyperref}
\usepackage{fontawesome}
\usepackage{mdframed}
\usepackage[svgnames,dvipsnames,usenames]{xcolor}
\usepackage{listings}
\usepackage[ampersand]{easylist}
\usepackage[top=0.5cm,left=0.5cm,right=0.5cm,bottom=0.5cm]{geometry}

\lstset{
    language=bash,
    frame=single,
    backgroundcolor=\color{white},
    basicstyle=\small,
    showspaces=false
}

\setlength{\parindent}{0pt}

\title{CMCC - Cheatsheet Bash}
\author{Fabio Viola}
\date{March 2020}

\renewcommand{\familydefault}{\sfdefault}

\begin{document}

\mdfsetup{
    roundcorner = 5pt,
    frametitlerule=true,
    frametitlebackgroundcolor=Black!80,
    frametitlefont={\bfseries\color{white}},
    backgroundcolor=Black!15!white,
    linecolor=Black!80!white,
    linewidth=1pt,
    innerbottommargin=10pt,
    skipbelow=1pt,
    innerbottommargin=1pt
}

\LARGE \textbf{Bash quick reference} \normalsize  


%%%%%%%%%%%%%%%%%%%%%%%%%%%%%%%%%%
%%
%% VARIABLES
%%
%%%%%%%%%%%%%%%%%%%%%%%%%%%%%%%%%%

\begin{mdframed}[frametitle=Variables]
Create a variable:
\begin{lstlisting}
VAR1="value"             # global
local VAR2="other value" # local
\end{lstlisting}
Arrays:
\begin{lstlisting}
MYLIST[0]="value1"      # on the fly
${MYLIST[i]}            # access element i
${MYLIST[*]}            # access whole list
${#MYLIST[*]}           # size of the list
$MYLIST+="value"        # add an element
unset $MYLIST[i]        # remove element i
\end{lstlisting}
\end{mdframed}

%%%%%%%%%%%%%%%%%%%%%%%%%%%%%%%%%%
%%
%% ARITHMETIC
%%
%%%%%%%%%%%%%%%%%%%%%%%%%%%%%%%%%%

\begin{mdframed}[frametitle=Arithmetic]
Simple arithmetic with \texttt{let} (output is on a variable):
\begin{lstlisting}
let a=2+2
let a++
\end{lstlisting}

Arithmetic with \texttt{expr} (output is on stdout):
\begin{lstlisting}
expr 2 + 2
expr 2 \* 2 # note the escape char!
VAR2=$(expr 2 - $VAR1) # assignment
\end{lstlisting}
\end{mdframed}

%%%%%%%%%%%%%%%%%%%%%%%%%%%%%%%%%%
%%
%% TEST CONDITIONS
%%
%%%%%%%%%%%%%%%%%%%%%%%%%%%%%%%%%%

\begin{mdframed}[frametitle=Test command for conditions]

Logic comparison:
\begin{lstlisting}
[ CONDITION1 -a CONDITION2 ] # and
[ CONDITION1 -o CONDITION2 ] # or
[ ! CONDITION ] # not
\end{lstlisting}

String comparison:
\begin{lstlisting}
[ STRING1 = STRING2 ]   # equal
[ STRING1 != STRING2 ]  # different
\end{lstlisting}

Integer comparison:
\begin{lstlisting}
[ INTEGER1 -eq INTEGER2 ]   # equal
[ INTEGER1 -ne INTEGER2 ]   # not equal
[ INTEGER1 -ge INTEGER2 ]   # greater/equal
[ INTEGER1 -gt INTEGER2 ]   # greater than
[ INTEGER1 -le INTEGER2 ]   # less/equal
[ INTEGER1 -lt INTEGER2 ]   # less than
\end{lstlisting}

File comparison:
\begin{lstlisting}
[ -f FILE ] # file exists and is a regular file
[ -d FILE ] # file exists and is a directory
\end{lstlisting}

\faInfoCircle\ For more type \texttt{man test}.

\vspace{3pt}
\end{mdframed}

%%%%%%%%%%%%%%%%%%%%%%%%%%%%%%%%%%
%%
%% CONDITIONS
%%
%%%%%%%%%%%%%%%%%%%%%%%%%%%%%%%%%%

\begin{mdframed}[frametitle=Conditions]
If-then-else:
\begin{lstlisting}
if CONDITION; 
    then ... ; 
    elif CONDITION ; then ... ; 
    else ... ; 
fi
\end{lstlisting}

Switch case construct:
\begin{lstlisting}
case EXPRESSION in
    VALUE1 )
        ... ;;
    VALUE2 )
        ... ;;
    ...
esac
\end{lstlisting}

\end{mdframed}

%%%%%%%%%%%%%%%%%%%%%%%%%%%%%%%%%%
%%
%% LOOPS
%%
%%%%%%%%%%%%%%%%%%%%%%%%%%%%%%%%%%

\begin{mdframed}[frametitle=Loops]
For loop:
\begin{lstlisting}
for VARIABLE in RANGE; do ... ; done
\end{lstlisting}

While loop:
\begin{lstlisting}
while CONDITION; do ... ; done
\end{lstlisting}
Until loop:
\begin{lstlisting}
until CONDITION; do ... ; done
\end{lstlisting}

\end{mdframed}

%%%%%%%%%%%%%%%%%%%%%%%%%%%%%%%%%%
%%
%% FUNCTIONS
%%
%%%%%%%%%%%%%%%%%%%%%%%%%%%%%%%%%%

\begin{mdframed}[frametitle=Functions]
Declare a function:
\begin{lstlisting}
functionName () { ...; }
\end{lstlisting}

Invoke a function:
\begin{lstlisting}
functionName ARG1 ARG2 ...
\end{lstlisting}

Arguments:
\begin{lstlisting}
$0         # function name
$#         # number of arguments provided
$* / $@    # list of all the arguments provided
"$@"       # as above, with separated items
\end{lstlisting}
\end{mdframed}


%%%%%%%%%%%%%%%%%%%%%%%%%%%%%%%%%%
%%
%% RANGES
%%
%%%%%%%%%%%%%%%%%%%%%%%%%%%%%%%%%%

\begin{mdframed}[frametitle=Ranges]
Two ways:
\begin{lstlisting}
{START..END}
seq START INCREMENT END
\end{lstlisting}

Examples with for loop:
\begin{lstlisting}
for I in {1..3} ; do echo $I ; done
for I in $(seq 0 1 10) ; do echo $I ; done
\end{lstlisting}
\end{mdframed}


%%%%%%%%%%%%%%%%%%%%%%%%%%%%%%%%%%
%%
%% SUBSHELLS
%%
%%%%%%%%%%%%%%%%%%%%%%%%%%%%%%%%%%

\begin{mdframed}[frametitle=Subshell]
Assign the output of a command to a variable:
\begin{lstlisting}
VAR1=$(command)
VAR2=`command`
\end{lstlisting}
\end{mdframed}


%%%%%%%%%%%%%%%%%%%%%%%%%%%%%%%%%%%%%%%%%%%%
%%
%% REFERENCES
%%
%%%%%%%%%%%%%%%%%%%%%%%%%%%%%%%%%%%%%%%%%%%%

\begin{mdframed}[frametitle=Resources]
\begin{lstlisting}
$ man bash
\end{lstlisting}
\end{mdframed}    

\end{document}
