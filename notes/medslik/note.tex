\section{Notes on the Users' manual}

MEDSLIK-II was originally conceived to carry out deterministic, short term simulations supporting oil spill emergency situations.

Unlike observed in the previous versions, MEDSLIK-II v2.0 supports \textbf{one single type of ocean fields} and \textbf{one type of atmospheric fields}.

Two Python scripts were prepared:

\begin{itemize}
    \item \texttt{preproc\_currents\_mdk2.py}: to translate MERCATOR currents' and SSTfields;
    \item \texttt{preproc\_winds\_mdk2.py}: and ERA-Interim atmospheric fields.
\end{itemize}  

Ocean fields are divided into three different files:

\begin{itemize}
    \item \textbf{meridional current}: \texttt{MDK\_ocean\_yymmdd\_V.nc}
    \item \textbf{zonal current}: \texttt{MDK\_ocean\_yymmdd\_U.nc}
    \item \textbf{sea surface temperature}: \texttt{MDK\_ocean\_yymmdd\_T.nc}
\end{itemize}

All of these files are generated by \texttt{preproc\_currents\_mdk2.py}. The user is expected to deliver currents in four different depths: 0m, 10m, 30m and 120m. 

Then, we have \textbf{wind fields}. Wind fields are given using a single file containing meridional and zonal 10m winds in m/s. Currently, the model ingests hourly resolution wind fields. These files are generated by \texttt{preproc\_winds\_mdk2.py}.


\

So, \texttt{preproc\_currents\_mdk2.py} takes an input folder containing CMEMS-GLO files and outputs three type of netCDF4 files, one for meridional current, another for zonal current and the last for sea surface temperature. The \texttt{preproc\_winds\_mdk2.py} also takes as input CMEMS-GLO files, but outputs one single type of netCDF4 file.