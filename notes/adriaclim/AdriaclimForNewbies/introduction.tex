\section{Introduction}

\begin{frame}{Adriaclim}
    
\textbf{Adriaclim} is an EU interreg project between Italy and Croatia aiming at:

\begin{itemize}
    \item enhancing CC adaptation capacity in coastal areas developing homogeneous and comparable data

    \item improving knowledge, capacity and cooperation on climate change observing and modeling systems

    \item developing advanced information system, tools and indicators for optimal CC adaptation planning 
\end{itemize}
    
\end{frame}

\begin{frame}{Our task}
    
Among the many tasks assigned to OPA, one that is particularly relevant is the creation of a coupled model through the CIME platform developed by NCAR.

This activity, carried out together with CMCC's REMHI division, is currently focused on the setup of a coupled model between an oceanographic component (i.e., NEMO) and an atmospheric one (i.e., WRF). In the next months, other models will be coupled\dots

\end{frame}

\section{CIME}

\begin{frame}{Introduction}

CIME contains:

\begin{itemize}
    \item the support scripts (configure, build, run, test)
    \item data models
    \item essential utility libraries
    \item a main
    \item other tools
\end{itemize}

to build a \textbf{single-executable coupled Earth System Model}. CIME is available in a stand-alone package that can be compiled and tested without active prognostic components but is typically included in the source of a climate model. CIME does not contain: any active components, any intra-component coupling capability (such as atmosphere physics-dynamics coupling).
    
\end{frame}