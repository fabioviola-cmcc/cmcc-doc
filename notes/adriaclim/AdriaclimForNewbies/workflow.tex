\section{Workflow}

\begin{frame}{Workflow}

The main steps to create and run a coupled model with CIME are:

\begin{enumerate}
    \item Creation of a case    
    \item Setting up the case
    \item Building the case
    \item Submitting the case
\end{enumerate}    

\ 

In the following slides, we will analyse in detail each of these steps.

\end{frame}


%%%%%%%%%%%%%%%%%%%%%%%%%%%%%%%%%%%%%%%%%%%%%%%%%%%%%%%%%%%%%%%%
%%
%% CREATION OF A CASE
%%
%%%%%%%%%%%%%%%%%%%%%%%%%%%%%%%%%%%%%%%%%%%%%%%%%%%%%%%%%%%%%%%%

\begin{frame}[fragile]{Creation of a case}
Create the new case:

\begin{verbatim}
$ ./scripts/create_newcase --compset C_NEMO --res T62_n13 \ 
  --case HELLOWORLD --mach zeus --run-unsupported
\end{verbatim}

\ 

The mandatory parameters are:

\begin{itemize}
    \item<1-> \textbf{compset}: a component set defining all the involved models. To have a list of the available compsets:
    \begin{verbatim}
$ ./scripts/query_config --compsets
    \end{verbatim}
    \item<2-> \textbf{case name}: a friendly name for the case. 
    
    \item<3-> \textbf{grid}: a string identifying a combination of the grids used by the models. A list of the available grids can be retrieved with: 
    \begin{verbatim}
$ ./scripts/query_config --grids
    \end{verbatim}
\end{itemize}
\end{frame}

%%%%%%%%%%%%%%%%%%%%%%%%%%%%%%%%%%%%%%%%%%%%%%%%%%%%%%%%%%%%%%%%
%%
%% CASE details
%%
%%%%%%%%%%%%%%%%%%%%%%%%%%%%%%%%%%%%%%%%%%%%%%%%%%%%%%%%%%%%%%%%

\begin{frame}{Details\dots}

\begin{itemize}
    \item<1-> Compset C\_NEMO is defined as:
    \begin{itemize}
        \item \texttt{2000\_DATM\%NYF\_SLND\_DICE\%SSMI\_NEMO\_DROF\%NYF\_SGLC\_SWAV}
        \item \dots that means NEMO as a prognostic model, atmosphere, ice and river as data models;
        \item The rest is defined as a set of stub models.
    \end{itemize}
    \item<2-> The grid is defined as:
    \begin{itemize}
        \item T62 for atmosphere and land, tn1v3 for ocean and ice
    \end{itemize}
\end{itemize}
    
\end{frame}


%%%%%%%%%%%%%%%%%%%%%%%%%%%%%%%%%%%%%%%%%%%%%%%%%%%%%%%%%%%%%%%%
%%
%% CASE SETUP
%%
%%%%%%%%%%%%%%%%%%%%%%%%%%%%%%%%%%%%%%%%%%%%%%%%%%%%%%%%%%%%%%%%

\begin{frame}[fragile]{Setting up the case}
To setup the case, let's move to the case directory and:

\begin{verbatim}
$ ./case.setup
\end{verbatim}

\ 

This scripts creates some additional directories, configuration files and scripts.

\end{frame}


%%%%%%%%%%%%%%%%%%%%%%%%%%%%%%%%%%%%%%%%%%%%%%%%%%%%%%%%%%%%%%%%
%%
%% CASE BUILD
%%
%%%%%%%%%%%%%%%%%%%%%%%%%%%%%%%%%%%%%%%%%%%%%%%%%%%%%%%%%%%%%%%%

\begin{frame}[fragile]{Building the case}

Now it's time to build the model, so (still from the case root):

\begin{verbatim}
$ ./case.build
\end{verbatim}

\ 

If the process runs smoothly, at the end of the process we will have a single executable called \texttt{cesm.exe}.

\end{frame}



%%%%%%%%%%%%%%%%%%%%%%%%%%%%%%%%%%%%%%%%%%%%%%%%%%%%%%%%%%%%%%%%
%%
%% CASE RUN
%%
%%%%%%%%%%%%%%%%%%%%%%%%%%%%%%%%%%%%%%%%%%%%%%%%%%%%%%%%%%%%%%%%

\begin{frame}[fragile]{Running the case}

Once the model has been compiled, we can run it by submitting the request to the scheduler:

\begin{verbatim}
$ ./case.submit
\end{verbatim}

\ 

This process runs the coupled model and the following task called \texttt{archive}. We can check the status of the processes through \texttt{bjobs}.

\end{frame}



%%%%%%%%%%%%%%%%%%%%%%%%%%%%%%%%%%%%%%%%%%%%%%%%%%%%%%%%%%%%%%%%
%%
%% OPTIONAL STEPS
%%
%%%%%%%%%%%%%%%%%%%%%%%%%%%%%%%%%%%%%%%%%%%%%%%%%%%%%%%%%%%%%%%%

\begin{frame}[fragile]{Optional steps}

\begin{itemize}
    \item<1-> Modifying variables:
    \begin{verbatim}
$ ./xmlchange VARIABLE VALUE
    \end{verbatim}
    
    \item<2-> Modifying the namelists: we edit the namelists called \texttt{user\_nl\_\textit{xx}} where \texttt{\textit{xx}} is the model
    
    \item<3-> Previewing the namelists: 
    \begin{verbatim}
$ ./preview_namelists
    \end{verbatim}
    
    \item<4-> Previewing the run:
    \begin{verbatim}
$ ./preview_run
    \end{verbatim}

    \item<5-> Getting the list of components:
    \begin{verbatim}
$ ./query_config --components 
    \end{verbatim}

\end{itemize}
\end{frame}

