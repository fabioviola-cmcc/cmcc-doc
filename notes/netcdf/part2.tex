\documentclass[9pt]{beamer}
\usepackage[utf8]{inputenc}
\usetheme{Montpellier}
\usecolortheme{dolphin}
\usepackage{listings}
\usepackage{fontawesome}
\usepackage{hyperref}
\usepackage[ampersand]{easylist}

\title{Dealing with NetCDF files}
\subtitle{Training module 1\dots}
\date{2021/05/04}


\AtBeginSection[]
{
 \begin{frame}<beamer>
 \frametitle{Table of contents}
 \tableofcontents[currentsection]
 \end{frame}
}


\begin{document}

\author[Viola] % (optional, for multiple authors)
{F.~Viola\inst{1}}

\institute[CMCC] % (optional)
{
  \inst{1}%
  CMCC -- Centro EuroMediterraneo sui Cambiamenti Climatici\\
  OPA Division -- Ocean Predictions and Applications
}

\logo{\includegraphics[height=1.5cm]{logo.png}}

\maketitle

%%%%%%%%%%%%%%%%%%%%%%%%%%%%%%%%%%%%%%%%%%%%%%%%%%%%%%%%%%%%%%%%%%%%%%%%%%%%%%%%%%%%%%%%%%%%%%%%%%%
\begin{frame}[fragile]{Supported languages}

The NetCDF file format is highly supported. Among the many languages implementing NetCDF, we mention:

\begin{itemize}
    \item \textbf{Fortran}
    \item \textbf{Python}
    \item C/C++
    \item Java
    \item Ruby
    \item R
    \item Matlab
    \item IDL
    \item Perl
    \item Tcl/Tk
    \item Ada
\end{itemize}
\end{frame}

%%%%%%%%%%%%%%%%%%%%%%%%%%%%%%%%%%%%%%%%%%%%%%%%%%%%%%%%%%%%%%%%%%%%%%%%%%%%%%%%%%%%%%%%%%%%%%%%%%%
\begin{frame}[fragile]{Python}

In Python it is first advisable to create an environment with conda:

\begin{verbatim}
$ conda create -n envName
$ conda activate envName
\end{verbatim}

\pause

\ 

The netcdf library that we will use in Python is called \texttt{netCDF4}

\begin{verbatim}
$ conda install netcdf4
\end{verbatim}

\end{frame}

%%%%%%%%%%%%%%%%%%%%%%%%%%%%%%%%%%%%%%%%%%%%%%%%%%%%%%%%%%%%%%%%%%%%%%%%%%%%%%%%%%%%%%%%%%%%%%%%%%%
\begin{frame}[fragile]{Python (1)}

First of all import the library:

\begin{verbatim}
from netCDF4 import Dataset
\end{verbatim}

\pause

\ 

Then, we can open NetCDF files with:

\begin{verbatim}
ds = Dataset(filename, "r", "NETCDF4")
\end{verbatim}
\end{frame}


%%%%%%%%%%%%%%%%%%%%%%%%%%%%%%%%%%%%%%%%%%%%%%%%%%%%%%%%%%%%%%%%%%%%%%%%%%%%%%%%%%%%%%%%%%%%%%%%%%%
\begin{frame}[fragile]{Python (2)}

\begin{itemize}
    \item \texttt{ds} -- shows several information
    \item \texttt{ds.dimensions} -- returns a dictionary of all the dimensions
    \item \texttt{ds.variables} -- shows a dictionary of all the variables
    \item \texttt{ds["varname"]} -- shows info about the variable \textit{varname}
    \item \texttt{ds["varname"][:]} -- access data according to numpy rules (see \textit{slicing})
\end{itemize}

\end{frame}

%%%%%%%%%%%%%%%%%%%%%%%%%%%%%%%%%%%%%%%%%%%%%%%%%%%%%%%%%%%%%%%%%%%%%%%%%%%%%%%%%%%%%%%%%%%%%%%%%%%
\begin{frame}[fragile]{Python (3)}

\begin{itemize}
    \item \texttt{dim1 = Dataset.createDimension(...)} -- to create a dimension
    \item \texttt{var1 = Dataset.createVariable(...)} -- to create a variable
    \item \texttt{var1[:] = data} -- to assign data to the variable
\end{itemize}

\end{frame}


%%%%%%%%%%%%%%%%%%%%%%%%%%%%%%%%%%%%%%%%%%%%%%%%%%%%%%%%%%%%%%%%%%%%%%%%%%%%%%%%%%%%%%%%%%%%%%%%%%%
\begin{frame}[fragile]{Python (4)}

The library NetCDF4 is quite useful in many cases, but you will soon feel limited using it\dots\ Then, the natural evolution will be to use \texttt{xarray}.

\end{frame}


\end{document}
